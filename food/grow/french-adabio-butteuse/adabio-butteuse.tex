%% LyX 2.0.2 created this file.  For more info, see http://www.lyx.org/.
%% Do not edit unless you really know what you are doing.
\documentclass[twoside,twocolumn,french]{paper}
\renewcommand{\familydefault}{\sfdefault}
\usepackage[T1]{fontenc}
\usepackage[utf8]{inputenc}
\usepackage[a4paper]{geometry}
\geometry{verbose,tmargin=2cm,bmargin=2cm,lmargin=2cm,rmargin=2cm}
\pagestyle{plain}
\setcounter{tocdepth}{2}
\setlength{\parskip}{\smallskipamount}
\setlength{\parindent}{0pt}
\usepackage{babel}
\addto\extrasfrench{%
   \providecommand{\og}{\leavevmode\flqq~}%
   \providecommand{\fg}{\ifdim\lastskip>\z@\unskip\fi~\frqq}%
}

\usepackage[unicode=true,pdfusetitle,
 bookmarks=true,bookmarksnumbered=false,bookmarksopen=false,
 breaklinks=false,pdfborder={0 0 1},backref=section,colorlinks=false]
 {hyperref}

\makeatletter
%%%%%%%%%%%%%%%%%%%%%%%%%%%%%% User specified LaTeX commands.
\usepackage{babel}
\setlength{\columnsep}{1.3cm}

\makeatother

\begin{document}

\title{Butteuse à planche}


\institution{ADABio-Autoconstruction}

\maketitle
\href{http://www.adabio-autoconstruction.org/outils/tous-les-outils/butteuse-a-planche.html}{URL}
\begin{abstract}
Cet outil remplace la charrue pour l’enfouissement des déchets de
cultures et des engrais verts. Il permet aussi de relever une butte
aplanie par les passages d’outils à dents, type herse étrille ou Vibroplanche.
\end{abstract}
Le modèle de Butteuse à trois paires de disques a été préféré pour
limiter la profondeur de travail tout en permettant le façonnage d’une
butte d’au moins 40 cm de hauteur. Les disques ainsi répartis couvrent
facilement toute la surface d’une planche pour un travail superficiel
(destruction d’engrais verts jeunes ou d’adventices). Les matières
organiques sont enfouies dans le volume de la butte, permettant une
dégradation optimale, contrairement à un enfouissement en fond de
labour. Des conditions sèches sont en revanche un handicap pour l’évolution
de la matière organique. Mais les conséquences sont moins graves et
les solutions plus faciles. Les modèles de Butteuse à asperges du
marché ont été adaptés pour répondre aux exigences du travail en planches
permanentes. Ont ainsi été rajoutés : des pattes-d’oie sur dents double
spire pour le binage des allées, des roues de jauge pour le contrôle
de la profondeur, un buttoir central pour la reprise de buttes déjà
formées et un triangle d’attelage, type Accord, pour faciliter l’accrochage
et le décrochage de l’outil.

La Butteuse ainsi équipée permet un gain de temps appréciable par
rapport à une charrue à deux ou trois socs, généralement utilisée
en maraîchage. Elle travaille une largeur de 1,90 m à une vitesse
de 3 ou 4 km/heure. De plus, dès qu’une planche est libérée en milieu
de parcelle, celle-ci peut être buttée sans gêne, permettant une maîtrise
de l’enherbement éventuel.

D’autre part, les disques en rotation s’usent beaucoup moins que les
pièces fixes d’une charrue exposées aux frottements, d’où un coût
d’entretien beaucoup plus faible.


\section*{Les limites}

Il est nécessaire d’effectuer un broyage avant le passage de la Butteuse
afin de faciliter l’incorporation des végétaux par les disques. La
conduite de l’outil est délicate, surtout en situation de dévers.
Les roues de râteau-faneur utilisées en roues de jauge réduisent beaucoup
ce problème en maintenant une symétrie dans la profondeur de travail
entre les deux côtés de l’outil.


\section*{Liens}

plans dans le \href{http://www.adabio-autoconstruction.org/le-livre/le-livre.html}{guide}
à acheter

\href{http://www.adabio-autoconstruction.org/IMG/jpg/dsc_1220.jpg}{photo 1}

\href{http://www.adabio-autoconstruction.org/IMG/jpg/dsc_1236.jpg}{photo 2}

\href{http://www.adabio-autoconstruction.org/IMG/jpg/dsc_1248.jpg}{photo 3}

\href{http://www.adabio-autoconstruction.org/IMG/jpg/dsc03431.jpg}{photo 4}

\href{http://www.adabio-autoconstruction.org/IMG/jpg/dsc03434.jpg}{photo 5}

\href{http://www.dailymotion.com/video/xs3hqf_butteuse-jt_lifestyle\#.UPQwBWfs-ZQ}{Jardins du Temple}

\href{http://www.dailymotion.com/video/xwtcou_butteuse-a-planche-adabioautoconstruction_tech\#.UP1Y3Gfs-ZQ}{Jardins de la Vallée chez les frères Ravard}

\href{http://forum.adabio-autoconstruction.org/viewforum.php?f=37}{forum}
\end{document}
