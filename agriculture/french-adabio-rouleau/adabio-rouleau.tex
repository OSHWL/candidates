%% LyX 2.0.2 created this file.  For more info, see http://www.lyx.org/.
%% Do not edit unless you really know what you are doing.
\documentclass[twoside,twocolumn,french]{paper}
\renewcommand{\familydefault}{\sfdefault}
\usepackage[T1]{fontenc}
\usepackage[utf8]{inputenc}
\usepackage[a4paper]{geometry}
\geometry{verbose,tmargin=2cm,bmargin=2cm,lmargin=2cm,rmargin=2cm}
\pagestyle{plain}
\setcounter{tocdepth}{2}
\setlength{\parskip}{\smallskipamount}
\setlength{\parindent}{0pt}
\usepackage{babel}
\addto\extrasfrench{%
   \providecommand{\og}{\leavevmode\flqq~}%
   \providecommand{\fg}{\ifdim\lastskip>\z@\unskip\fi~\frqq}%
}

\usepackage[unicode=true,pdfusetitle,
 bookmarks=true,bookmarksnumbered=false,bookmarksopen=false,
 breaklinks=false,pdfborder={0 0 1},backref=section,colorlinks=false]
 {hyperref}

\makeatletter
%%%%%%%%%%%%%%%%%%%%%%%%%%%%%% User specified LaTeX commands.
\usepackage{babel}
\setlength{\columnsep}{1.3cm}

\makeatother

\begin{document}

\title{Rouleau perceur}


\institution{ADABio-Autoconstruction}

\maketitle
\href{http://www.adabio-autoconstruction.org/outils/tous-les-outils/rouleau-perceur.html}{URL}
\begin{abstract}
Le rouleau perceur est un outil qui permet de percer le paillage plastique
ou de tracer en terre nue pour pouvoir semer ou transplanter. Cet
outil est très modulable. Il est ainsi possible de choisir le nombre
de rangs et l’écartement entre les plants.
\end{abstract}
Il est constitué d’un châssis auquel est fixé un rouleau muni de plusieurs
roues sur lesquelles sont vissées des pointes. L’utilisateur tire
l’outil en marchant le long de la planche. Il est également possible
de le pousser pour assurer une meilleure pénétration des éléments
perceurs.


\section*{Détails des équipements}


\paragraph*{Le châssis}

Ce genre de pièce se réalise facilement avec du matériel de récupération
tels des restes de structures de serre. Le choix de profilés est donné
à titre indicatif. La dimension des profilés importe peu, les efforts
en présence n’étant pas très importants. L’assemblage de tubes de
serre peut se faire de plusieurs façons. Il est souvent utile de façonner
les tubes au marteau et à l’aide d’un étau pour augmenter les surfaces
de contact. Une soudure peut être réalisée mais il est délicat de
faire du travail très propre. Il est également possible d’écraser
complètement le tube et de le percer pour un assemblage par vis ou
rivets, mais ce montage est moins solide car le profil n’a plus sa
résistance structurelle.


\paragraph*{L’axe}

Sa forme, un profilé carré de 40 mm, permet l’entraînement des roues
qui viendront se glisser dessus. La partie centrale des roues est
un profilé carré de 45 mm de large pour 2 mm d’épaisseur, laissant
un jeu fonctionnel de 1 mm.


\paragraph*{Les roues}

Les roues ont un diamètre de 410 mm et s’obtiennent en cintrant un
plat de 60 x 5. Le perçage de ce plat ne doit pas être effectué avant
le cintrage sous peine d’obtenir une roue bosselée. Quarante perçages
espacés de 32 mm laissent une grande liberté pour régler l’espacement.
La roue est bloquée sur l’axe dans la position voulue à l’aide d’une
grosse vis papillon. Le rouleau peut être utilisé avec deux, trois,
quatre ou cinq roues. Autre cas : une seule roue est équipée de pointes
alors que deux autres, laissées lisses, servent à stabiliser le rouleau
(courges, courgettes, par exemple).


\paragraph*{Les pointes}

Elles sont réalisées en soudant dos à dos deux cornières biseautées
au préalable. De une à huit pointes peuvent être montées sur chaque
roue.


\section*{Les limites}

Le faible poids de cet outil ne permet pas toujours un perçage optimum.
Il faut alors alourdir la structure, mais cela demandera plus d’énergie
à l’utilisateur pour tirer le rouleau. Pousser au lieu de tirer améliore
le travail tout en offrant une meilleure visibilité. 

Par ailleurs, la forme du guidon permet d’opérer en bordure de serre
sans être gêné par les parois.


\section*{Liens}

plans dans le \href{http://www.adabio-autoconstruction.org/le-livre/le-livre.html}{guide}
à acheter

\href{http://www.adabio-autoconstruction.org/IMG/jpg/photo_rouleau_perceur_6_.jpg}{photo 1}

\href{http://www.adabio-autoconstruction.org/IMG/jpg/l1010307.jpg}{photo 2}
\end{document}
