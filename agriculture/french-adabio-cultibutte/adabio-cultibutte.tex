%% LyX 2.0.2 created this file.  For more info, see http://www.lyx.org/.
%% Do not edit unless you really know what you are doing.
\documentclass[twoside,twocolumn,french]{paper}
\renewcommand{\familydefault}{\sfdefault}
\usepackage[T1]{fontenc}
\usepackage[utf8]{inputenc}
\usepackage[a4paper]{geometry}
\geometry{verbose,tmargin=2cm,bmargin=2cm,lmargin=2cm,rmargin=2cm}
\pagestyle{plain}
\setcounter{tocdepth}{2}
\setlength{\parskip}{\smallskipamount}
\setlength{\parindent}{0pt}
\usepackage{babel}
\addto\extrasfrench{%
   \providecommand{\og}{\leavevmode\flqq~}%
   \providecommand{\fg}{\ifdim\lastskip>\z@\unskip\fi~\frqq}%
}

\usepackage[unicode=true,pdfusetitle,
 bookmarks=true,bookmarksnumbered=false,bookmarksopen=false,
 breaklinks=false,pdfborder={0 0 1},backref=section,colorlinks=false]
 {hyperref}

\makeatletter
%%%%%%%%%%%%%%%%%%%%%%%%%%%%%% User specified LaTeX commands.
\usepackage{babel}
\setlength{\columnsep}{1.3cm}

\makeatother

\begin{document}

\title{Cultibutte}


\institution{ADABio-Autoconstruction}

\maketitle
\href{http://www.adabio-autoconstruction.org/outils/tous-les-outils/cultibutte-des-jardins-du-temple.html}{URL}
\begin{abstract}
Le Cultibutte a été baptisé ainsi car les principaux organes de travail
sont des dents de cultivateur complétées par une paire de disques.
Il permet le travail en butte et en planche permanente pour la reprise
d’un labour ou d’une fin de culture. Il est conçu pour façonner ou
entretenir les buttes.
\end{abstract}
Ses dents ne permettent toutefois pas l’enfouissement des végétaux.
Ceux-ci doivent être broyés et/ou mixés avec la fraise pour faciliter
l’incorporation. Mais leur évolution dans le sol s’en trouve facilitée,
car elle se fait dans le volume de la butte en milieu aéré et bien
drainé. L’amplitude de travail de l’outil est de 0 à 30 cm de profondeur
par rapport au niveau des allées. Ce réglage s’effectue facilement
en ajustant la hauteur des roues de jauge. Aux Jardins du Temple,
cet outil est réglé pour travailler à 10 cm maximum sous le niveau
des allées. Dans une terre préservée, comme ce peut être le cas en
planches permanentes, le décompactage profond n’est pas indispensable.
Ce respect du sol induit également des économies d’usure des socs,
d’énergie et de temps par la vitesse ainsi autorisée (2 à 4 km/heure).
Le rendement du chantier est ainsi multiplié par cinq ou six, comparativement
à la rotobêche. De plus, le résultat est nettement meilleur car la
structure du sol n’est pas brisée artificiellement, limitant ainsi
les phénomènes de battance ou de prise en masse. Avec des socs appropriés
et des changements de réglages, le Cultibutte peut aussi déchaumer,
sarcler et même éventuellement décompacter. Il est également possible
de travailler plus superficiellement grâce à l’articulation du châssis
tout en conservant la forme de la butte. Et avec l’option des socs
démontables, des pattes-d’oie plus larges peuvent être installées
rapidement pour détruire un engrais vert jeune ou une levée importante
d’adventices.


\section*{Les limites}

Le Cultibutte a tendance à « bourrer » si les déchets de culture ou
d’engrais vert sont trop importants. Le broyage et/ou le mixage sont
indispensables pour permettre un passage ultérieur, une fois la végétation
décomposée. L’utilisation de l’outil avec seulement trois dents limite
le problème. La terre est légèrement déplacée ce qui oblige à alterner
les sens de passage pour peaufiner le travail en bout de planche.
La lutte contre les vivaces à rhizomes n’est pas optimale en bout
de planche car les outils à dents nécessitent quelques mètres pour
atteindre leur position de travail.


\section*{Les adaptations possibles}

Le bâti ouvert sur les côtés permet l’ajout d’une ou deux dents latérales
équipées de pattes-d’oie pour le sarclage des allées. Il est facile
et pratique de construire des chapes de fixation des dents par emboîtement
et des broches pour un démontage rapide quand deux ou trois dents
suffisent à faire un bon travail. L’effort de traction est ainsi réduit,
l’usure des outils et le brassage excessif du sol également.


\section*{Liens}

plans dans le \href{http://www.adabio-autoconstruction.org/le-livre/le-livre.html}{guide}
à acheter

\href{http://www.adabio-autoconstruction.org/IMG/jpg/dsc03425.jpg}{photo 1}

\href{http://www.adabio-autoconstruction.org/IMG/jpg/dsc03391.jpg}{photo 2}

\href{http://www.adabio-autoconstruction.org/IMG/jpg/dsc03395.jpg}{photo 3}

\href{http://www.adabio-autoconstruction.org/IMG/jpg/p1110750.jpg}{photo 4}

\href{http://www.adabio-autoconstruction.org/IMG/jpg/copie_de_p1110795.jpg}{photo 5}

\href{http://forum.adabio-autoconstruction.org/viewforum.php?f=38}{forum}


\paragraph*{Vidéos\protect \\
}

\href{http://www.dailymotion.com/video/xtbmw5_cultibutte-en-action_tech\#.UPAMcmfs-ZQ}{cultibutte en action}

\href{http://www.dailymotion.com/video/xtbdvn_cultibutte-changement-de-soc_tech\#.UPAMyGfs-ZQ}{changement de soc}

\href{http://www.dailymotion.com/video/xtbdqq_reglages-cultibutte_tech\#.UPAM-Wfs-ZQ}{réglages}
\end{document}
