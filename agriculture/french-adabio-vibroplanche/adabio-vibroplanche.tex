%% LyX 2.0.2 created this file.  For more info, see http://www.lyx.org/.
%% Do not edit unless you really know what you are doing.
\documentclass[twoside,twocolumn,french]{paper}
\renewcommand{\familydefault}{\sfdefault}
\usepackage[T1]{fontenc}
\usepackage[utf8]{inputenc}
\usepackage[a4paper]{geometry}
\geometry{verbose,tmargin=2cm,bmargin=2cm,lmargin=2cm,rmargin=2cm}
\pagestyle{plain}
\setcounter{tocdepth}{2}
\setlength{\parskip}{\smallskipamount}
\setlength{\parindent}{0pt}
\usepackage{babel}
\addto\extrasfrench{%
   \providecommand{\og}{\leavevmode\flqq~}%
   \providecommand{\fg}{\ifdim\lastskip>\z@\unskip\fi~\frqq}%
}

\usepackage[unicode=true,pdfusetitle,
 bookmarks=true,bookmarksnumbered=false,bookmarksopen=false,
 breaklinks=false,pdfborder={0 0 1},backref=section,colorlinks=false]
 {hyperref}

\makeatletter
%%%%%%%%%%%%%%%%%%%%%%%%%%%%%% User specified LaTeX commands.
\usepackage{babel}
\setlength{\columnsep}{1.3cm}

\makeatother

\begin{document}

\title{Vibroplanche}


\institution{ADABio-Autoconstruction}

\maketitle
\href{http://www.adabio-autoconstruction.org/outils/tous-les-outils/le-vibroplanche-des-jardins-du.html}{URL}
\begin{abstract}
Comme vous l’aurez deviné, son nom découle de l’usage de dents de
Vibroculteur pour sa conception. C’est donc un outil d’affinage du
sol, amélioré par l’ajout d’une herse étrille réglable et d’un rouleau
plombeur hydrauliquement assisté.
\end{abstract}
Sa vocation est de remplacer partiellement les outils rotatifs de
type Cultirateau pour les préparations finales avant semis ou plantations.
Sa vitesse de travail est alors de 2 ou 3 km/h. De plus, le Vibroplanche
est très utile pour la maîtrise de l’herbe sur les planches en attente
de culture. Dans ce cas le travail est très superficiel et la vitesse
permise et efficace sera de 5 à 6 km/h. Et là encore cet outil se
démarque des engins rotatifs par sa sobriété. De même, la destruction
d’un engrais vert jeune est tout à fait possible suivie d’un passage
de Butteuse pour assurer l’enfouissement. Les lumières sur les éléments
de liaison et les articulations permettent de faciliter la conduite
et les réglages selon les travaux à effectuer.


\section*{Les limites}

Sa limite est l’enfouissement des résidus de cultures qui ne sont
pas aussi bien incorporés qu’avec un outil rotatif de type Cultirateau.
Les dents traînent les végétaux trop longs et l’outil finit par bourrer.


\section*{Les adaptations possibles}

On peut remplacer le rouleau lisse par un rouleau cage, ce qui permet
d’affiner la surface sans la lisser. Le support rouleau doit être
légèrement modifié pour cette option.

Il est possible de remplacer les dents classiques de Vibroculteur
par des dents droites type “efface trace de semoir” qui remontent
moins la terre et les graines d’adventices. Il serait intéressant
de pouvoir régler la hauteur de travail des dents de sarclage des
allées pour s’adapter à toutes les situations.


\section*{Liens}

plans dans le \href{http://www.adabio-autoconstruction.org/le-livre/le-livre.html}{guide}
à acheter

\href{http://www.adabio-autoconstruction.org/IMG/jpg/dscn0289.jpg}{photo 1}

\href{http://www.adabio-autoconstruction.org/IMG/jpg/dsc_1211.jpg}{photo 2}

\href{http://www.adabio-autoconstruction.org/IMG/jpg/p1110761.jpg}{photo 3}

\href{http://www.adabio-autoconstruction.org/IMG/jpg/dsc_1266.jpg}{photo 4}

\href{http://www.adabio-autoconstruction.org/IMG/jpg/p1110767.jpg}{photo 5}

\href{http://www.adabio-autoconstruction.org/IMG/jpg/p1110760.jpg}{photo 6}

\href{http://www.adabio-autoconstruction.org/IMG/jpg/dsc_1205.jpg}{photo 7}

\href{http://www.adabio-autoconstruction.org/IMG/jpg/dscn0280.jpg}{photo 8}

\href{http://www.dailymotion.com/video/xt76lr_vibroplanche_tech\#.UPAK6mfs-ZQ}{Jardins du Temple}

\href{http://www.dailymotion.com/video/xwtdez_vibroplanche-adabio-autoconstruction_tech\#.UP1XPmfs-ZQ}{Jardins de la Vallée chez les frères Ravard}

\href{http://forum.adabio-autoconstruction.org/viewforum.php?f=39}{forum}
\end{document}
